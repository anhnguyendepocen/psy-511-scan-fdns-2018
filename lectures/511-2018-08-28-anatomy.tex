\documentclass[]{article}
\usepackage{lmodern}
\usepackage{amssymb,amsmath}
\usepackage{ifxetex,ifluatex}
\usepackage{fixltx2e} % provides \textsubscript
\ifnum 0\ifxetex 1\fi\ifluatex 1\fi=0 % if pdftex
  \usepackage[T1]{fontenc}
  \usepackage[utf8]{inputenc}
\else % if luatex or xelatex
  \ifxetex
    \usepackage{mathspec}
  \else
    \usepackage{fontspec}
  \fi
  \defaultfontfeatures{Ligatures=TeX,Scale=MatchLowercase}
\fi
% use upquote if available, for straight quotes in verbatim environments
\IfFileExists{upquote.sty}{\usepackage{upquote}}{}
% use microtype if available
\IfFileExists{microtype.sty}{%
\usepackage{microtype}
\UseMicrotypeSet[protrusion]{basicmath} % disable protrusion for tt fonts
}{}
\usepackage[margin=1in]{geometry}
\usepackage{hyperref}
\hypersetup{unicode=true,
            pdftitle={511-2018-08-28-anatomy},
            pdfauthor={Rick Gilmore},
            pdfborder={0 0 0},
            breaklinks=true}
\urlstyle{same}  % don't use monospace font for urls
\usepackage{longtable,booktabs}
\usepackage{graphicx,grffile}
\makeatletter
\def\maxwidth{\ifdim\Gin@nat@width>\linewidth\linewidth\else\Gin@nat@width\fi}
\def\maxheight{\ifdim\Gin@nat@height>\textheight\textheight\else\Gin@nat@height\fi}
\makeatother
% Scale images if necessary, so that they will not overflow the page
% margins by default, and it is still possible to overwrite the defaults
% using explicit options in \includegraphics[width, height, ...]{}
\setkeys{Gin}{width=\maxwidth,height=\maxheight,keepaspectratio}
\IfFileExists{parskip.sty}{%
\usepackage{parskip}
}{% else
\setlength{\parindent}{0pt}
\setlength{\parskip}{6pt plus 2pt minus 1pt}
}
\setlength{\emergencystretch}{3em}  % prevent overfull lines
\providecommand{\tightlist}{%
  \setlength{\itemsep}{0pt}\setlength{\parskip}{0pt}}
\setcounter{secnumdepth}{0}
% Redefines (sub)paragraphs to behave more like sections
\ifx\paragraph\undefined\else
\let\oldparagraph\paragraph
\renewcommand{\paragraph}[1]{\oldparagraph{#1}\mbox{}}
\fi
\ifx\subparagraph\undefined\else
\let\oldsubparagraph\subparagraph
\renewcommand{\subparagraph}[1]{\oldsubparagraph{#1}\mbox{}}
\fi

%%% Use protect on footnotes to avoid problems with footnotes in titles
\let\rmarkdownfootnote\footnote%
\def\footnote{\protect\rmarkdownfootnote}

%%% Change title format to be more compact
\usepackage{titling}

% Create subtitle command for use in maketitle
\newcommand{\subtitle}[1]{
  \posttitle{
    \begin{center}\large#1\end{center}
    }
}

\setlength{\droptitle}{-2em}

  \title{511-2018-08-28-anatomy}
    \pretitle{\vspace{\droptitle}\centering\huge}
  \posttitle{\par}
    \author{Rick Gilmore}
    \preauthor{\centering\large\emph}
  \postauthor{\par}
      \predate{\centering\large\emph}
  \postdate{\par}
    \date{2018-08-28 08:23:20}


\begin{document}
\maketitle

{
\setcounter{tocdepth}{1}
\tableofcontents
}
\subsection{Prelude}\label{prelude}

\url{https://www.youtube.com/snO68aJTOpM}

\subsection{Today's Topics}\label{todays-topics}

\begin{itemize}
\tightlist
\item
  Wrap-up on functional methods
\item
  Gross neuroanatomy
\end{itemize}

\section{Wrap-up on functional
methods}\label{wrap-up-on-functional-methods}

\subsection{Stimulating the brain}\label{stimulating-the-brain}

\begin{itemize}
\tightlist
\item
  Pharmacological
\item
  Electrical (\textbf{Transcranial Direct Current Stimulation -
  \href{https://en.wikipedia.org/wiki/Transcranial_direct-current_stimulation}{tDCS}})
\item
  Magnetic (\textbf{Transcranial magnetic stimulation -
  \href{https://en.wikipedia.org/wiki/Transcranial_magnetic_stimulation}{TMS}})
\end{itemize}

\begin{center}\rule{0.5\linewidth}{\linethickness}\end{center}

\begin{center}\rule{0.5\linewidth}{\linethickness}\end{center}

\begin{figure}

{\centering \includegraphics{https://upload.wikimedia.org/wikipedia/commons/6/67/Transcranial_magnetic_stimulation} 

}

\caption{TMS}\label{fig:unnamed-chunk-1}
\end{figure}

\subsection{Stimulating the brain}\label{stimulating-the-brain-1}

\begin{itemize}
\tightlist
\item
  Spatial/temporal resolution?
\item
  Assume stimulation mimics natural activity?
\end{itemize}

\subsection{Deep brain stimulation as
therapy}\label{deep-brain-stimulation-as-therapy}

\begin{itemize}
\tightlist
\item
  Depression
\item
  Epilepsy
\item
  Parkinson's Disease
\end{itemize}

\subsection{}\label{section}

\url{http://www.nimh.nih.gov/images/health-and-outreach/mental-health-topic-brain-stimulation-therapies/dbs_60715_3.jpg}

\begin{center}\rule{0.5\linewidth}{\linethickness}\end{center}

\url{https://youtu.be/KDjWdtDyz5I}

\subsection{Optogenetics more closely mimics brain
activity}\label{optogenetics-more-closely-mimics-brain-activity}

\subsection{\texorpdfstring{\href{https://en.wikipedia.org/wiki/Optogenetics}{Optogenetics}}{Optogenetics}}\label{optogenetics}

\begin{itemize}
\tightlist
\item
  Gene splicing techniques insert light-sensitive molecules into
  neuronal membranes
\item
  Application of light at specific wavelengths alters neuronal function
\item
  Cell-type specific and temporally precise control
\end{itemize}

\begin{center}\rule{0.5\linewidth}{\linethickness}\end{center}

\url{https://youtu.be/FlGbznBmx8M}

\subsection{\texorpdfstring{\emph{Sim}ulating the
brain}{Simulating the brain}}\label{simulating-the-brain}

\begin{itemize}
\tightlist
\item
  Computer/mathematical models of brain function
\item
  Example: neural networks
\item
  Cheap, noninvasive, can be stimulated or ``lesioned''
\end{itemize}

\begin{center}\rule{0.5\linewidth}{\linethickness}\end{center}

Blue Brain project

\href{http://doi.org/10.1038/nrn1848}{Markram, 2006}

\begin{center}\rule{0.5\linewidth}{\linethickness}\end{center}

\subsection{Main points}\label{main-points}

\begin{itemize}
\tightlist
\item
  Multiple structural, functional methods
\item
  Different levels of spatial \& temporal analysis
\item
  Functional tools have different strengths \& weaknesses
\end{itemize}

\section{Gross neuroanatomy}\label{gross-neuroanatomy}

\begin{center}\rule{0.5\linewidth}{\linethickness}\end{center}


\url{https://www.pastmedicalhistory.co.uk/the-nervous-system-of-harriet-cole/}

\subsection{Brain anatomy through
dance}\label{brain-anatomy-through-dance}

 Your browser does not support the audio tag.

\subsection{Finding our way around}\label{finding-our-way-around}

\subsubsection{Anterior/Posterior}\label{anteriorposterior}

\subsubsection{Medial/Lateral}\label{mediallateral}

\subsubsection{Superior/Inferior}\label{superiorinferior}

\subsubsection{Dorsal/Ventral}\label{dorsalventral}

\subsubsection{Rostral/Caudal}\label{rostralcaudal}

\subsection{Directional image}\label{directional-image}

\url{https://upload.wikimedia.org/wikipedia/commons/thumb/e/e7/Blausen_0019_AnatomicalDirectionalReferences.png/800px-Blausen_0019_AnatomicalDirectionalReferences.png}

\subsection{Bipeds vs.~quadripeds}\label{bipeds-vs.quadripeds}

\url{https://upload.wikimedia.org/wikipedia/commons/thumb/0/00/1303_Human_Neuroaxis.jpg/800px-1303_Human_Neuroaxis.jpg}

\subsection{No matter how you slice
it}\label{no-matter-how-you-slice-it}

\subsubsection{Horizontal/Axial}\label{horizontalaxial}

\subsubsection{Coronal/Transverse/Frontal}\label{coronaltransversefrontal}

\subsubsection{Sagittal (from the side)}\label{sagittal-from-the-side}

\subsection{Slice diagram}\label{slice-diagram}

\url{http://www.scienceteacherprogram.org/biology/chillemistudentguide1-06/brain_directions_planes_sections__directions_-_small.gif}

\subsection{Supporting structures}\label{supporting-structures}

\subsubsection{Meninges}\label{meninges}

\subsubsection{Ventricular system}\label{ventricular-system}

\subsubsection{Blood supply}\label{blood-supply}

\subsection{Meninges}\label{meninges-1}

\subsubsection{\texorpdfstring{Dura mater (`tough
mother')}{Dura mater (tough mother)}}\label{dura-mater-tough-mother}

\subsubsection{Arachnoid membrane}\label{arachnoid-membrane}

\subsubsection{Subarachnoid space}\label{subarachnoid-space}

\subsubsection{\texorpdfstring{Pia mater (`gentle
mother')}{Pia mater (gentle mother)}}\label{pia-mater-gentle-mother}

\subsection{Meninges}\label{meninges-2}

\url{https://upload.wikimedia.org/wikipedia/commons/thumb/8/8e/Meninges-en.svg/1280px-Meninges-en.svg.png}

\subsection{Ventricular system}\label{ventricular-system-1}

\url{https://upload.wikimedia.org/wikipedia/commons/d/d4/Blausen_0896_Ventricles_Brain.png}

\subsection{Ventricles}\label{ventricles}

\subsubsection{Lateral (1st \& 2nd)}\label{lateral-1st-2nd}

\subsubsection{3rd}\label{rd}

\subsubsection{Cerebral aqueduct}\label{cerebral-aqueduct}

\subsubsection{4th}\label{th}

\subsubsection{Cerebrospinal fluid (CSF)}\label{cerebrospinal-fluid-csf}

\begin{itemize}
\tightlist
\item
  Clears metabolites during sleep (Xie et al., 2013).
\end{itemize}

\subsection{Blood Supply}\label{blood-supply-1}

\url{http://surgery.med.miami.edu/images/Circulation_of_brain.gif}

\subsection{Blood Supply}\label{blood-supply-2}

\subsubsection{Arteries}\label{arteries}

\begin{itemize}
\tightlist
\item
  Circle of Willis
\end{itemize}

\subsubsection{Blood/brain barrier}\label{bloodbrain-barrier}

\begin{itemize}
\tightlist
\item
  Cells forming blood vessel walls tightly packed
\item
  Active transport of molecules typically required
\end{itemize}

\subsection{Blood/brain barrier}\label{bloodbrain-barrier-1}

\url{http://www.nature.com/nrn/journal/v7/n1/images/nrn1824-f3.jpg}

\subsection{Area Postrema}\label{area-postrema}

\begin{itemize}
\tightlist
\item
  Brainstem, blood-brain barrier thin
\end{itemize}

\url{http://www.nature.com/nrendo/journal/v9/n10/images/nrendo.2013.136-f2.jpg}

\subsection{Organization of the Nervous
System}\label{organization-of-the-nervous-system}

\subsubsection{Central Nervous System
(CNS)}\label{central-nervous-system-cns}

\begin{itemize}
\tightlist
\item
  Brain
\item
  Spinal Cord
\item
  Everything encased in bone
\end{itemize}

\subsubsection{Peripheral Nervous System
(PNS)}\label{peripheral-nervous-system-pns}

\subsection{Organization of the brain}\label{organization-of-the-brain}

\begin{longtable}[]{@{}llll@{}}
\toprule
Major division & Ventricular Landmark & Embryonic Division &
Structure\tabularnewline
\midrule
\endhead
Forebrain & Lateral & Telencephalon & Cerebral cortex\tabularnewline
& & & Basal ganglia\tabularnewline
& & & Hippocampus, amygdala\tabularnewline
& Third & Diencephalon & Thalamus\tabularnewline
& & & Hypothalamus\tabularnewline
Midbrain & Cerebral Aqueduct & Mesencephalon & Tectum,
tegmentum\tabularnewline
\bottomrule
\end{longtable}

\subsection{Organization of the
brain}\label{organization-of-the-brain-1}

\begin{longtable}[]{@{}llll@{}}
\toprule
Major division & Ventricular Landmark & Embryonic Division &
Structure\tabularnewline
\midrule
\endhead
Hindbrain & 4th & Metencephalon & Cerebellum, pons\tabularnewline
& -- & Mylencephalon & Medulla oblongata\tabularnewline
\bottomrule
\end{longtable}

\begin{center}\rule{0.5\linewidth}{\linethickness}\end{center}

\subsection{\texorpdfstring{\href{https://en.wikipedia.org/wiki/File:EmbryonicBrain.svg}{Hindbrain}}{Hindbrain}}\label{hindbrain}

\subsection{\texorpdfstring{\href{https://en.wikipedia.org/wiki/File:EmbryonicBrain.svg}{Hindbrain}}{Hindbrain}}\label{hindbrain-1}

\begin{itemize}
\tightlist
\item
  Structures adjacent to 4th ventricle

  \begin{itemize}
  \tightlist
  \item
    Medulla oblongata
  \item
    Cerebellum
  \item
    Pons
  \end{itemize}
\end{itemize}

\begin{center}\rule{0.5\linewidth}{\linethickness}\end{center}

\begin{center}\rule{0.5\linewidth}{\linethickness}\end{center}

\subsection{\texorpdfstring{\href{https://en.wikipedia.org/wiki/Medulla_oblongata}{Medulla
oblongata}}{Medulla oblongata}}\label{medulla-oblongata}

\url{https://upload.wikimedia.org/wikipedia/commons/6/69/1311_Brain_Stem.jpg}

\subsection{Medulla}\label{medulla}

\begin{itemize}
\tightlist
\item
  Cardiovascular regulation
\item
  Muscle tone
\item
  Fibers of passage

  \begin{itemize}
  \tightlist
  \item
    Ascending fibers (from body), a.k.a. afferents
  \item
    Descending fibers (\textbf{e}xiting brain), a.k.a.,
    \textbf{e}fferents
  \end{itemize}
\end{itemize}

\subsection{\texorpdfstring{\href{https://en.wikipedia.org/wiki/Cerebellum}{Cerebellum}}{Cerebellum}}\label{cerebellum}

\begin{itemize}
\tightlist
\item
  ``Little brain''
\item
  Dorsal to pons
\item
  Movement coordination, simple learning
\end{itemize}

\begin{center}\rule{0.5\linewidth}{\linethickness}\end{center}

\begin{center}\rule{0.5\linewidth}{\linethickness}\end{center}

\subsection{\texorpdfstring{\href{https://en.wikipedia.org/wiki/Pons}{Pons}}{Pons}}\label{pons}

\begin{itemize}
\tightlist
\item
  Bulge on brain stem
\item
  Neuromodulatory nuclei
\item
  Relay to cerebellum
\end{itemize}

\subsection{Midbrain}\label{midbrain}

\url{http://antranik.org/wp-content/uploads/2011/11/the-brain-stem-mid-brain-left-lateral-view-superior-colliculus-inferior-cerebellar-peduncle.jpg}

\subsection{Midbrain components}\label{midbrain-components}

\subsubsection{Tectum}\label{tectum}

\subsubsection{Tegmentum}\label{tegmentum}

\subsection{Tectum}\label{tectum-1}

\url{https://upload.wikimedia.org/wikipedia/commons/0/0b/Gray719.png}

\begin{center}\rule{0.5\linewidth}{\linethickness}\end{center}

\subsection{\texorpdfstring{\href{https://en.wikipedia.org/wiki/Tectum}{Tectum}}{Tectum}}\label{tectum-2}

\begin{itemize}
\tightlist
\item
  ``Roof'' of the midbrain
\item
  Superior and inferior colliculus
\item
  Reflexive orienting of eyes, head, ears
\end{itemize}

\subsection{\texorpdfstring{\href{https://en.wikipedia.org/wiki/Tegmentum}{Tegmentum}}{Tegmentum}}\label{tegmentum-1}

\begin{itemize}
\tightlist
\item
  ``Floor'' of the midbrain
\item
  Species-typical movement sequences
\item
  Neuromodulatory nuclei

  \begin{itemize}
  \tightlist
  \item
    Norepinephrine (NE)
  \item
    Serotonin (5-HT)
  \item
    Dopamine (DA) -- from \emph{ventral tegmental area (VTA)}
  \end{itemize}
\end{itemize}

\begin{center}\rule{0.5\linewidth}{\linethickness}\end{center}

\subsection{Forebrain}\label{forebrain}

\url{http://classconnection.s3.amazonaws.com/252/flashcards/1048252/png/forebrain1328987872235.png}

\subsection{Forebrain Components}\label{forebrain-components}

\subsubsection{Diencephalon}\label{diencephalon}

\subsubsection{Telencephalon}\label{telencephalon}

\subsection{\texorpdfstring{\href{https://en.wikipedia.org/wiki/Diencephalon}{Diencephalon}}{Diencephalon}}\label{diencephalon-1}

\url{https://upload.wikimedia.org/wikipedia/commons/a/a0/1310_Diencephalon.jpg}

\subsection{Diencephalon Components}\label{diencephalon-components}

\begin{itemize}
\tightlist
\item
  Thalamus
\item
  Hypothalamus
\end{itemize}

\begin{center}\rule{0.5\linewidth}{\linethickness}\end{center}

\subsection{\texorpdfstring{\href{https://en.wikipedia.org/wiki/Thalamus}{Thalamus}}{Thalamus}}\label{thalamus}

\url{http://neurobiologychapter3.weebly.com/uploads/1/4/1/8/1418733/5118342.jpg?401x231}

\subsection{Thalamus functions}\label{thalamus-functions}

\begin{itemize}
\tightlist
\item
  Input to cortex
\item
  Functionally distinct \emph{nuclei} (collection of neurons)

  \begin{itemize}
  \tightlist
  \item
    Lateral geniculate nucleus (LGN), vision
  \item
    Medial geniculate nucleus (MGN), audition
  \item
    Pulvinar, attention?
  \end{itemize}
\end{itemize}

\subsection{\texorpdfstring{\href{https://en.wikipedia.org/wiki/Hypothalamus}{Hypothalamus}}{Hypothalamus}}\label{hypothalamus}

\begin{itemize}
\tightlist
\item
  Five Fs: fighting, fleeing/freezing, feeding, and reproduction
\item
  Controls pituitary gland (``master'' gland)

  \begin{itemize}
  \tightlist
  \item
    Anterior pituitary (indirect release of hormones)
  \item
    Posterior pituitary (direct release of hormones)

    \begin{itemize}
    \tightlist
    \item
      Oxytocin
    \item
      Vasopressin
    \end{itemize}
  \end{itemize}
\end{itemize}

\subsection{Hypothalamus}\label{hypothalamus-1}

\url{http://higheredbcs.wiley.com/legacy/college/tortora/0470565101/hearthis_ill/pap13e_ch14_illustr_audio_mp3_am/simulations/figures/hypothalamus.jpg}

\begin{center}\rule{0.5\linewidth}{\linethickness}\end{center}

\subsection{Telencephalon}\label{telencephalon-1}

\begin{itemize}
\tightlist
\item
  Basal ganglia
\item
  Hippocampus, amygdala
\item
  Cerebral cortex
\end{itemize}

\subsection{\texorpdfstring{\href{https://en.wikipedia.org/wiki/Basal_ganglia}{Basal
Ganglia}}{Basal Ganglia}}\label{basal-ganglia}

\begin{itemize}
\tightlist
\item
  Skill and habit learning
\item
  Linked to Tourette syndrome, obsessive-compulsive disorder (OCD),
  addiction, movement disorders
\item
  Example: Parkinson's Disease
\end{itemize}

\subsection{Basal ganglia}\label{basal-ganglia-1}

\url{http://webspace.ship.edu/cgboer/basalgangliagray.gif}

\subsection{Basal ganglia}\label{basal-ganglia-2}

\begin{itemize}
\tightlist
\item
  Striatum

  \begin{itemize}
  \tightlist
  \item
    Caudate nucleus
  \item
    Putamen
  \end{itemize}
\item
  Globus pallidus
\item
  Subthalamic nucleus
\item
  Substantia nigra (tegmentum)
\end{itemize}

\begin{center}\rule{0.5\linewidth}{\linethickness}\end{center}

\begin{center}\rule{0.5\linewidth}{\linethickness}\end{center}

\subsection{Hippocampus}\label{hippocampus}

\begin{itemize}
\tightlist
\item
  Immediately lateral to lateral ventricles
\item
  Memories of specific facts or events
\item
  Fornix projects to hypothalamus
\item
  Mammillary bodies
\end{itemize}

\subsection{Hippocampus}\label{hippocampus-1}

\url{http://homepage.smc.edu/wissmann_paul/physnet/anatomynet/anatomy/amy.jpg}

\begin{center}\rule{0.5\linewidth}{\linethickness}\end{center}

\begin{center}\rule{0.5\linewidth}{\linethickness}\end{center}

\subsection{\texorpdfstring{Amygdala
(``almond'')}{Amygdala (almond)}}\label{amygdala-almond}

\begin{itemize}
\tightlist
\item
  Physiological state, behavioral readiness, affect
\item
  NOT the fear center! (LeDoux, 2015).
\end{itemize}

\subsection{Amygdala}\label{amygdala}

\url{http://homepage.smc.edu/wissmann_paul/physnet/anatomynet/anatomy/amy.jpg}

\begin{center}\rule{0.5\linewidth}{\linethickness}\end{center}

\begin{center}\rule{0.5\linewidth}{\linethickness}\end{center}

\subsection{Next time}\label{next-time}

\begin{itemize}
\tightlist
\item
  Gilmore's cautionary notes
\item
  The cerebral cortex
\item
  The peripheral nervous system
\end{itemize}

\subsection*{References}\label{references}
\addcontentsline{toc}{subsection}{References}

\hypertarget{refs}{}
\hypertarget{ref-ledoux_amygdala_2015}{}
LeDoux, J. (2015, August 10). The Amygdala Is NOT the Brain's Fear
Center. \emph{Psychology Today}. Retrieved from
\url{https://www.psychologytoday.com/blog/i-got-mind-tell-you/201508/the-amygdala-is-not-the-brains-fear-center}

\hypertarget{ref-xie2013sleep}{}
Xie, L., Kang, H., Xu, Q., Chen, M. J., Liao, Y., Thiyagarajan, M.,
\ldots{} others. (2013). Sleep drives metabolite clearance from the
adult brain. \emph{Science}, \emph{342}(6156), 373--377.
\url{https://doi.org/10.1126/science.1241224}


\end{document}
